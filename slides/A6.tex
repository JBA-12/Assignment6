\documentclass{beamer}

\usepackage{xspace}

 \usepackage{calc}                                             %%
    \usepackage{multirow}                                         %%
    \usepackage{hhline}                                           %%
    \usepackage{ifthen}   
    
\newcounter{saveenumi}
\newcommand{\seti}{\setcounter{saveenumi}{\value{enumi}}}
\newcommand{\conti}{\setcounter{enumi}{\value{saveenumi}}}                                      

\setbeamertemplate{caption}[numbered]{}
    
\providecommand{\pr}[1]{\ensuremath{\Pr\left\{#1\right\}}}
\providecommand{\brak}[1]{\ensuremath{\left(#1\right)}}
\providecommand{\cbrak}[1]{\ensuremath{\left\{#1\right\}}}
\providecommand{\sbrak}[1]{\ensuremath{{}\left[#1\right]}}


% Theme choice:
\usetheme{CambridgeUS}

% Title page details: 
\title{ASSIGNMENT 6 : CBSE PROBABILITY CLASS- 12  EXAMPLE - 25} 
\author{AI21BTECH11016}
\date{\today}
\logo{\large \LaTeX{}}


\begin{document}

% Title page frame
\begin{frame}
    \titlepage 
\end{frame}

% Remove logo from the next slides
\logo{}

% Outline frame
\begin{frame}{Outline}
    \tableofcontents
\end{frame}

% Outline frame
\begin{frame}{Question}
\begin{block}{}
Find the probability distribution of number of doublets in three throws of
a pair of dice.
\end{block}
\end{frame}

\section{Declaration of Random Variables}
\begin{frame}{Solution}
Let X=\cbrak{0,1,2,3} be a random variable representing the number of doublets.\\

\begin{table}[ht!]
    \centering
    \input{tables/table6(1).tex}
    \caption{Random Variables}
    \label{Tab:Table1}
\end{table}
\end{frame}

\section{Finding Probabilities from Binomial Distribution}

\begin{frame}
\text For a single throw the possible doublets are :
\begin{center}
\text (1,1) , (2,2), (3,3), (4,4), (5,5), (6,6)\\
\end{center}
\item $\Rightarrow$ Probability of getting a doublet = $\frac{1}{6}$\\
Let \textbf{p} be the probability of getting a doublet in one throw and Let \textbf{q} be the probability of not getting a doublet in one throw.\\ 
\centering
\begin{align}
\Rightarrow p & = \frac{1}{6}\\
q & = \frac{5}{6}
\end{align}
\end{frame}

\begin{frame}
\textbf{From the binomial distribution :}\\
\begin{enumerate}
\item {
\begin{align}
\pr{X = 0} & = {3 \choose 0} \times p^0 \times q^3\\ & = 1 \times \sbrak{\frac{1}{6}}^0 \times \sbrak{\frac{5}{6}}^3\\ & = \frac{125}{216}
\end{align}}
\item {
\begin{align}
\pr{X = 1} & = {3 \choose 1} \times p^1 \times q^2\\ & = 3 \times \sbrak{\frac{1}{6}}^1 \times \sbrak{\frac{5}{6}}^2\\ & = 3 \times \frac{25}{216}\\ & = \frac{75}{216}
\end{align}}
\seti
\end{enumerate}
\end{frame}

\begin{frame}
\begin{enumerate}
\conti
\item{
\begin{align}
\pr{X = 2} & = {3 \choose 2} \times p^2 \times q^1\\ & = {3 \choose 2} \times \sbrak{\frac{1}{6}}^2 \times \sbrak{\frac{5}{6}}^1\\ & = 3 \times \frac{5}{216}\\ & = \frac{15}{216}
\end{align}}
\item{ 
\begin{align}
\pr{X = 3} & = {3 \choose 3} \times p^3 \times q^0\\ & = {3 \choose 3} \times \sbrak{\frac{1}{6}}^3 \times \sbrak{\frac{5}{6}}^0\\ & = \frac{1}{216}
\end{align}}
\end{enumerae}
\end{frame}


\section{Probability Distribution Table}

\begin{frame}{Probability Distribution Table}
$\Rightarrow$ The Probability distribution of number of doublets in three throws of
a pair of dice is :
    \begin{block}{}
     \begin{table}[ht!]
    \centering
    \input{tables/table6(2).tex}
    \caption{Probability Distribution}
    \label{Tab:2}
\end{table}   
    \end{block}
\end{frame} 


\end{document}
