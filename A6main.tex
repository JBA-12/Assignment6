\let\negmedspace\undefined
\let\negthickspace\undefined
\RequirePackage{amsmath}
\documentclass[journal,12pt,twocolumn]{IEEEtran}
 \usepackage{gensymb}
 \usepackage{graphicx}
 \usepackage{polynom}
\usepackage{amssymb}
\usepackage{amsthm}
 \usepackage{stfloats}
 \usepackage{bm}
\usepackage{enumitem}
 \usepackage{mathtools}
 \usepackage{tikz}
\usepackage[breaklinks=true]{hyperref}
\usepackage{listings}
    \usepackage{color}                                            %%
    \usepackage{array}                                            %%
                                            %%
    \usepackage{calc}                                             %%
    \usepackage{multirow}                                         %%
    \usepackage{hhline}                                           %%
    \usepackage{ifthen}                                           %%
  %optionally (for landscape tables embedded in another document): %%
    \usepackage{lscape}     
    \usepackage{amsmath}

\DeclareMathOperator*{\Res}{Res}
\DeclareMathOperator*{\equals}{=}



\hyphenation{op-tical net-works semi-conduc-tor}
                                %%

\lstset{
%language=C,
frame=single, 
breaklines=true,
columns=fullflexible
}

\begin{document}

\newtheorem{theorem}{Theorem}[section]
\newtheorem{problem}{Problem}
\newtheorem{proposition}{Proposition}[section]
\newtheorem{lemma}{Lemma}[section]
\newtheorem{corollary}[theorem]{Corollary}
\newtheorem{example}{Example}[section]
\newtheorem{definition}[problem]{Definition}
\newcommand{\BEQA}{\begin{eqnarray}}
\newcommand{\EEQA}{\end{eqnarray}}
\newcommand{\define}{\stackrel{\triangle}{=}}
\newcommand*\circled[1]{\tikz[baseline=(char.base)]{
    \node[shape=circle,draw,inner sep=2pt] (char) {#1};}}
\bibliographystyle{IEEEtran}
%\bibliographystyle{ieeetr}
\providecommand{\mbf}{\mathbf}
\providecommand{\pr}[1]{\ensuremath{\Pr\left\{#1\right\}}}
\providecommand{\qfunc}[1]{\ensuremath{Q\left(#1\right)}}
\providecommand{\sbrak}[1]{\ensuremath{{}\left[#1\right]}}
\providecommand{\lsbrak}[1]{\ensuremath{{}\left[#1\right.}}
\providecommand{\rsbrak}[1]{\ensuremath{{}\left.#1\right]}}
\providecommand{\brak}[1]{\ensuremath{\left(#1\right)}}
\providecommand{\lbrak}[1]{\ensuremath{\left(#1\right.}}
\providecommand{\rbrak}[1]{\ensuremath{\left.#1\right)}}
\providecommand{\cbrak}[1]{\ensuremath{\left\{#1\right\}}}
\providecommand{\lcbrak}[1]{\ensuremath{\left\{#1\right.}}
\providecommand{\rcbrak}[1]{\ensuremath{\left.#1\right\}}}
\theoremstyle{remark}
\newtheorem{rem}{Remark}
\newcommand{\sgn}{\mathop{\mathrm{sgn}}}
\providecommand{\abs}[1]{\left\vert#1\right\vert}
\providecommand{\res}[1]{\Res\displaylimits_{#1}} 
\providecommand{\norm}[1]{\left\lVert#1\right\rVert}
%\providecommand{\norm}[1]{\lVert#1\rVert}
\providecommand{\mtx}[1]{\mathbf{#1}}
\providecommand{\mean}[1]{E\left[ #1 \right]}
\providecommand{\fourier}{\overset{\mathcal{F}}{ \rightleftharpoons}}
%\providecommand{\hilbert}{\overset{\mathcal{H}}{ \rightleftharpoons}}
\providecommand{\system}{\overset{\mathcal{H}}{ \longleftrightarrow}}
	%\newcommand{\solution}[2]{\textbf{Solution:}{#1}}
\newcommand{\solution}{\noindent \textbf{Solution: }}
\newcommand{\question}{\noindent \textbf{Question: }}
\newcommand{\cosec}{\,\text{cosec}\,}
\providecommand{\dec}[2]{\ensuremath{\overset{#1}{\underset{#2}{\gtrless}}}}
\newcommand{\myvec}[1]{\ensuremath{\begin{pmatrix}#1\end{pmatrix}}}
\newcommand{\mydet}[1]{\ensuremath{\begin{vmatrix}#1\end{vmatrix}}}

\makeatletter
\@addtoreset{figure}{problem}
\makeatother
\let\StandardTheFigure\thefigure
\let\vec\mathbf
\def\putbox#1#2#3{\makebox[0in][l]{\makebox[#1][l]{}\raisebox{\baselineskip}[0in][0in]{\raisebox{#2}[0in][0in]{#3}}}}
     \def\rightbox#1{\makebox[0in][r]{#1}}
     \def\centbox#1{\makebox[0in]{#1}}
     \def\topbox#1{\raisebox{-\baselineskip}[0in][0in]{#1}}
     \def\midbox#1{\raisebox{-0.5\baselineskip}[0in][0in]{#1}}


\vspace{3cm}
       
\title{ASSIGNMENT 6 : CBSE PROBABILITY CLASS-12} 
\author{AI21BTECH11016} 
\date{May 2022}     


\maketitle

\newpage

\bigskip


\textbf{EXAMPLE - 25}\\

\question \\
Find the probability distribution of number of doublets in three throws of
a pair of dice.\\

\solution \\
Let X=\cbrak{0,1,2,3} be a random variable representing the number of doublets.\\\\

\begin{table}[ht!]
    \centering
    \input{tables/table6(1).tex}
    \caption{}
    \label{table:table1}
\end{table}

\text For a single throw the possible doublets are :
\begin{center}
\text (1,1) , (2,2), (3,3), (4,4), (5,5), (6,6)
\end{center}
$\Rightarrow$ Probability of getting a doublet = $\frac{1}{6}$ \\\\
\text From the binomial distribution :\\

\begin{enumerate}[label=(\roman*)]
\item {
\begin{align}
\pr{X = 0} & = {3 \choose 0} \times \sbrak{\frac{1}{6}}^0 \times \sbrak{\frac{5}{6}}^3\\ & = \frac{125}{216}
\end{align}}

\item {
\begin{align}
\pr{X = 1} & = {3 \choose 1} \times \sbrak{\frac{1}{6}}^1 \times \sbrak{\frac{5}{6}}^2\\ & = 3 \times \frac{25}{216}\\ & = \frac{75}{216}
\end{align}}

\item{
\begin{align}
\pr{X = 2} & = {3 \choose 2} \times \sbrak{\frac{1}{6}}^2 \times \sbrak{\frac{5}{6}}^1\\ & = 3 \times \frac{5}{216}\\ & = \frac{15}{216}
\end{align}}

\item{ 
\begin{align}
\pr{X = 3} & = {3 \choose 3} \times \sbrak{\frac{1}{6}}^3 \times \sbrak{\frac{5}{6}}^0\\ & = \frac{1}{216}
\end{align}}

\end{enumerate}

$\Rightarrow$ The Probability distribution of number of doublets in three throws of
a pair of dice is :

\begin{table}[ht!]
    \centering
    \input{tables/table6(2).tex}
    \caption{}
    \label{table:table2}
\end{table}

\end{document}